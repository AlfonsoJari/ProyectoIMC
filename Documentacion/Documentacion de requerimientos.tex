
\documentclass{article}
\usepackage{graphicx}
% Esto es para poder escribir acentos directamente:
\usepackage[latin1]{inputenc}
% Esto es para que el LaTeX sepa que el texto est� en espa�ol:
\usepackage[spanish]{babel}
% Paquetes de la AMS:
\usepackage{amsmath, amsthm, amsfonts}
\usepackage[left=3cm,right=3cm,top=3cm,bottom=3cm]{geometry} 


\usepackage{graphicx}

\def\RR{\mathbb{R}}
\def\ZZ{\mathbb{Z}}

\newcommand{\abs}[1]{\left\vert#1\right\vert}

\DeclareMathOperator{\Jac}{Jac}
\title{DOCUMENTACION DE REQUERIMIENTOS}
\author{ALEXI RAMIREZ RUIZ , ALFONSO JARI MAYA HERNANDEZ\\

	\\
  \LARGE Universidad Veracruzana\\
  
  \\
  
  \LARGE Ingenier�a de Software\\
  
  \\  
  
  \LARGE Pruebas de Software\\
    
  \\
  
  \LARGE Adolfo Centeno T�llez\\  
  
  

}
\begin{document}
\maketitle
\newpage

\title 
\Huge{INDICE GENERAL} 
\large \\


 1 INTRODUCCION...............3\\


 1.1 PROPOSITO...................3\\


 2 REQUERIMIENTOS FUNCIONALES.................3\\


 2.1 FUNCIONALIDAD DEL SISTEMA................3\\


 3 REQUISITOS NO FUNCIONALES.......................4\\


 3.1 DESCRIPCION DE REQUERIMIENTOS NO FUNCIONALES...............4\\


 3.2 INTERFAZ DEL SISTEMA..............5\\
 


\section{INTRODUCCION}
\vspace{0.7 cm}
\Large Esta calculadora tiene como prop�sito el darles a los usuarios informaci�n de su �ndice de Masa Corporal (IMC) para tener conocimiento de su estado de salud y en qu� secci�n de peso se encuentra la persona.\\
 
 
\subsection{REQUISITOS FUNCIONALES}\label{sec:nada}
\begin{verse}
\vspace{0.5 cm}
\LARGE 2.1 Descripcion de requerimientos no funcionales\\

*	Descripci�n de los datos a ser ingresados en el sistema.\\
*	Se deben hacer las operaciones del peso y altura.\\
*	Muestra de la tabla con la descripci�n del peso, edad altura y el IMC.\\
*	El sistema puede ser utilizado en cualquier navegador.\\
*	El sistema muestra opciones para los usuarios tanto para hombre como para mujer.\\
*	El sistema proporciona datos seguros.\\


\end{verse}

\subsubsection{REQUISITOS NO FUNCIONALES}\label{sec:nada2}
  
\begin{verse}
\Large 3.1 Descripci�n de requerimientos Funcionales\\
* El programa ser� desarrollado en angular 5 usando los lenguajes JavaScript HTML.\\
* Muestra de tabla de SEEDO\\
* El programa funcionara en los servidores firebase y el de clase de pruebas.\\
* El programa no tiene horarios de uso.\\
* Descripci�n de las operaciones a ser realizadas en pantalla\\
* No recopila informaci�n personal de los usuarios que ya lo hayan ocupado.\\
* Generaci�n de Ticket de resultados\\
* Mostrara un formulario simple\\
* El programa se actualiza al momento de salir.\\

\end{verse}

\subsubsection{REQUERIMIENTOS FUNCIONALES Y NO FUNCIONALES}
\begin{verse}
\Large Requerimientos funcionales:	\\


-Inicio de sesion con correo electronico\\


-mostrar un mapa con informacion sobre el covid19\\


-Que la base de datos guarde nombre de usuario y su contrase�a para el inicio de sesion\\


-Mostrar las estadisticas disponibles\\
	

-Mostrar cantidad de casos de contagio en el mapa\\


-Generar sin problemas un mapa con la informacion de la base de datos del covid19\\


Requerimientos no funcionales:\\


-Adaptabilidad (Debe admitir la capacidad de trabajar con diferentes almacenes de datos)\\

\end{verse}

\subsubsection{INTERFAZ DEL SISTEMA}
\begin{verse}

\begin{figure}[h]
     \includegraphics[scale=0.8]{IMAGEN1}
     \caption{INTERFAZ}  
\end{figure}
\end{verse}

\begin{figure}[h]
     \includegraphics[scale=0.8]{IMAGEN2}
     \caption{RESULTADOS}
\end{figure}       



\end{document}
